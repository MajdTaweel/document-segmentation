%! Author = Majd Taweel
%! Date = 30-Nov-20

% Preamble
\documentclass[conference]{IEEEtran}

% Packages
\usepackage{cite}
\usepackage{amsmath,amssymb,amsfonts}
\usepackage{algorithmic}
\usepackage{graphicx}
\usepackage{textcomp}
\usepackage{xcolor}
\usepackage{caption}
\usepackage{subcaption}
\usepackage{hyperref}
\def\BibTeX{{\rm B\kern-.05em{\sc i\kern-.025em b}\kern-.08em
    T\kern-.1667em\lower.7ex\hbox{E}\kern-.125emX}}

% Document
\begin{document}

    \title{
        \includegraphics[width=0.5\textwidth]{img/BZU-Logo.png}\\
        Document Layout Analysis\\
        \footnotesize Using Minimum Homogeneity Structure
    }

    \author{
        {Majd Taweel - 1161422}\\
        {Ibrahim Muala - 1160346}\\\\

        {Computer Vision}\\
        {Instructor: Aziz Qaroush}
    }
    \maketitle

    \begin{abstract}
    This report demonstrates our implementation of the Minimum Homogeneity Structure (MHS) Algorithm.
    It firstly states the problem encountered, the motivation around solving such problem and a brief description
    of the system proposed.
    Furthermore, it shows an overview of the proposed system.
    It further provides results achieved after implementing the proposed system.
    Finally, it explains the conclusions achieved from implementing this algorithm.
    \end{abstract}

    \begin{IEEEkeywords}
    Document layout analysis, Minimum homogeneity algorithm, Minimum homogeneity structure, OCR
    \end{IEEEkeywords}

    \section{Introduction}

    Document layout analysis is a rising topic in modern days.
    It has received more attention due to the need of exporting old printed documents into a digital version.
    This is important because organizations of all sizes and types need to have better and faster
    access to their previously recorded data, archives and reports that were printed on physical paper.
    These documents are firstly analysed using an accurate document analysis algorithm and the piped to an
    Optical Character Recognition (OCR) model to deduce the text content these documents contain.

    The MHS algorithm \cite{mhs}, analyzes documents on several different layers sequentially before classifying the
    objects contained in the document. Fig.~\ref{fig:mhs} provides a block diagram of the MHS algorithm.
    \begin{figure}[htbp]
        \centerline{\includegraphics[width=\linewidth]{img/mhs.jpg}}
        \caption{Block diagram of the MHS algorithm \cite{mhs}.}
        \label{fig:mhs}
    \end{figure}
    Our implementation first starts with pre-processing, outputting a binary (black and white) image.
    Then the binary image continues to text and non-text classification layer, which separates text and non-text
    content.
    This classification layer is actually what is known as Minimum Homogeneity Algorithm (MHA) \cite{mha}, the previous version
    of the MHS algorithm. Fig.~\ref{fig:mha} shows a flowchart of the classification layer (MHA).
    \begin{figure}[htbp]
        \centerline{\includegraphics[width=\linewidth]{img/mha.jpg}}
        \caption{Flowchart for the classification process (MHA) of the MHS system \cite{mhs}.}
        \label{fig:mha}
    \end{figure}
    This results into two different document for text and non-text content.
    Those two documents then passes different processes.
    Text lines are extracted from the text document and then paragraphs are separated using these extracted lines.
    Text regions are then deduced from the resulted paragraphs.
    The non-text document continues to yet another classification layer that then determines each non-text element's
    type (line, table, separator, image and negative-text).
    After that, the deduced regions from the text and non-text documents are refined and then labelled, resulting in
    the final page layout.

    The rest of the report is organized in the following manner, Sect.~\ref{sect:2} gives an overview of the system.
    Sect.~\ref{sect:3} demonstrates results of our implementation.
    Sect.~\ref{sect:4} shows the limitations of our algorithm.
    Finally, the conclusion is found in Sect.~\ref{sect:5}.


    \section{System Overview}\label{sect:2}

    The MHS algorithm consists of several processes or layers as stated before.
    These layers are explained in the following sections.

    \subsection{Pre-processing}

    This layer prepares the image for classification, and the quality of the operation it does on the image are reflected
    on the remainder of the layers.
    The binary image's quality greatly effects the rest of the operations.
    Take for instance, bla bla bla

    \begin{figure}[htbp]
        \centering
        \begin{subfigure}[b]{0.49\linewidth}
            \centering
            \includegraphics[width=\linewidth]{img/img.png}
            \caption{Original Image.}
            \label{img:org}
        \end{subfigure}
        \hfill
        \begin{subfigure}[b]{0.49\linewidth}
            \centering
            \includegraphics[width=\linewidth]{img/bin.png}
            \caption{Binary Image.}
            \label{img:bin}
        \end{subfigure}
        \caption{Image Pre-Processing and Binarization.}
        \label{fig:bin}
    \end{figure}

    The operations performed in this layer are as follows:
    \begin{enumerate}
        \item If the image is colored it is converted to a grayscale image.
        \item The image is slightly smoothed to remove some of the noise it may contain.
        \item The smoothed image is converted to a binary image.
        \item The binary image is resized to a certain resolution so that it does not take
        unreasonable computation time if its original resolution is very big.
    \end{enumerate}

    Fig.~\ref{fig:bin} shows a sample image and its respective binary image.

    % \subsection{Connected Components Analysis}

    \subsection{Heuristic Filter}

    The heuristic filter filters the image from obvious non-text elements depending on some heuristics, hence, the name
    heuristic filter.

    Specifically, there are four properties that a Connected Component (CC) is considered non-text if it suffice any of them:
    \begin{enumerate}
        \item The area of the CC is less or equals to 6 pixels (considered as noise).
        \item The bounding box of CC includes 4 or more other CCs.
        \item The density of the CC is less than 0.06.
        \item The height to width rate or width to height rate is less than 0.06.
    \end{enumerate}
    
    The non-text components determined from the previous step is then removed from the binary image, achieving a new image
    that almost only contains text.
    Actually, in \cite{mhs}, the authors stated that the fourth condition is removed due to the miss-classification of the Korean
    character (--).
    However, this condition is preserved in our implementation for two reasons, first, many characters and text
    elements will be classified as non-text throughout the layers, and we will mostly be dealing with English documents.
    This condition works well in identifying lines, removing it for the sake of one character isn't justified since many other text
    components will be miss-classified at first. Nonetheless, this miss-classification will be dealt with later.

    Fig.~\ref{img:heuristic-filter} shows the contours of text CCs (green) and non-text CCs (red) after going through the heuristic filter.
    It can be noticed as mentioned before, many text components we're classified as non-text.
    This inaccuracy must be dealt with either ways.

    \begin{figure}[htbp]
        \centerline{\includegraphics[width=0.5\linewidth]{img/heuristic-filter.png}}
        \caption{CCs Contours After Applying The Heuristic Filter.}
        \label{img:heuristic-filter}
    \end{figure}


    \subsection{Multilevel/Multi-layer Classification}

    This layer consists of two sub-layers, multilevel classification and multi-layer classification.
    It also contains a recursive filter that is applied on the regions extracted from both sub-layers.

    In multilevel classification, the image is split into homogeneous regions iteratively.
    This splitting is done both horizontally and vertically until all regions are homogeneous or can't
    be split anymore.

    The splitting is executed by leveraging horizontal and vertical porijections of images. In our
    implementation, a region is considered as an object, as soon as a region object is initialized,
    its projections are extracted. Fig.~\ref{img:projections} show such projections.

    \begin{figure}[htbp]
        \centerline{\includegraphics[width=\linewidth]{img/projections.png}}
        \caption{Horizontal and Vertical Projections and Bi-Level Projections of The Whole Image.}
        \label{img:projections}
    \end{figure}

    As we can notice from Fig.~\ref{img:projections}, the original projections is very rough and need to
    presented in a better way so we can benefit from them.
    That is why the bi-level projections is extracted by transforming each value that is more than 0 to 1.
    These projections represent horizontal and vertical lines in the image they were extracted from.
    Regions are split in regard with these lines. The splitting happens on what may call foreign lines.
    Those can be defined as white or black lines that have very different heights (widths in vertical projections)
    than its neighbors. Most likely, these lines have larger heights (widths in vertical projections). Such a line
    can be interpreted as headings (black line), or large whitespace (white line) separating chapters or paragraphs.

    After achieving the homogeneous regions, the recursive filter is applied on each of these regions.
    The filtered regions is further split into higher order homogeneous regions (if possible) and the
    recursive filter is applied again. This is repeated until the regions can't be modified anymore.
    Detailed specification of the recursive filter are provided in \cite{mha}.

    In multi-layer classification, the image is split into the first-level homogeneous regions (only once),
    then the recursive filter is applied on those regions. This operation is repeated until the regions
    can't be modified anymore.
    Note that, when repeating this operation, the first-level homogeneous regions are not further split,
    instead we re-split the whole image and retrieve the new first-level homogeneous regions.
    Fig.~\ref{fig:mll} illustrates the flow in the multilevel and multi-layer classification.

    \begin{figure}[htbp]
        \centerline{\includegraphics[width=\linewidth]{img/mll-flow.jpg}}
        \caption{Illustration of multilevel classification and multi-layer classification \cite{mhs}.}
        \label{fig:mll}
    \end{figure}

    The details of multilevel and multi-layer classification layer can be found in \cite{mhs} and \cite{mha}.

    Fig.~\ref{img:mll} shows text and non-text components after the multilevel and multi-layer classification.

    \begin{figure}[htbp]
        \centerline{\includegraphics[width=0.5\linewidth]{img/mll.png}}
        \caption{CCs Contours After Applying Multilevel and Multi-Layer Classification.}
        \label{img:mll}
    \end{figure}

    \subsection{Text Segmentation}

    After the multilevel and multi-layer classification, the text image is almost ready to be
    segmented into paragraphs.
    Now, this layer will double as a text segmenter and a text and non-text classifier.
    As we seen from the steps before and the images produced from them, some elements were
    miss-classified, but, these elements almost always reside inside a region that only contains
    elements from their correct class.

    Firstly, we perform morphological closing that only closes horizontally.
    Such closing can be achieved by using for example, a kernel of height 1 and width 5.
    This will connect horizontally close characters with each other.
    This can be seen in Fig.~\ref{img:morph-close-h}.
    
    \begin{figure}[htbp]
        \centerline{\includegraphics[width=0.5\linewidth]{img/morph-close.png}}
        \caption{Applying morphological closing on the text image.}
        \label{img:morph-close-h}
    \end{figure}
    
    Then, we filter the remaining whitespace depending on some heuristics as described in \cite{ws}.
    Fig.~\ref{img:ws-filter} is a color coded image that shows horizontally chains
    (blue, green and red) and removed whitespace between them
    (dark blue, dark green, dark red and dark purple).
    Dark blue represents removed whitespace due to its small size, dark green for being isolated,
    dark red for being within-column and dark purple for being labeled as a candidate within-column
    whitespace and resides at the top or bottom of a vertical whitespace chain.

    \begin{figure}[htbp]
        \centerline{\includegraphics[width=0.5\linewidth]{img/ws-filter.png}}
        \caption{Horizontal Chains and Removed Whitespace Between Them.}
        \label{img:ws-filter}
    \end{figure}

    After going through this process, almost all components belonging to the same line are connected.
    However, there is still one step before segmenting the paragraph.
    As noted many times before, some components were miss classified, to fix this we need to move text
    components classified as non-text to the text image and move the non-text components found in the
    text image to the non-text image.
    The way this is implemented is by checking for intersection between components.

    Firstly, the bounding box image is extracted from the text image.
    The bounding box image is simply an image that contains the bounding box of each CC instead of the CCs
    themselves. This is demonstrated in Fig.~\ref{img:bb}.

    \begin{figure}[htbp]
        \centerline{\includegraphics[width=0.5\linewidth]{img/bb.png}}
        \caption{Bounding box text image.}
        \label{img:bb}
    \end{figure}

    Then, morphological closing is applied to the bounding box image, this time vertically, using a kernel of
    height 3 and width 1 for example.
    In our implementation this was done for 4 iterations, to ensure that close components merge appropriately.
    Then we check for every element in the non-text document, if this element's area is less than some
    text element and they are almost completely intersected, then thus non-text element is a text element
    and belongs to the text document.
    The same thing is repeated for text elements.
    After these steps, the image in Fig.~\ref{img:intersections} is achieved.
    
    \begin{figure}[htbp]
        \centerline{\includegraphics[width=0.5\linewidth]{img/intersections.png}}
        \caption{Adding miss-classified text components to the text image.}
        \label{img:intersections}
    \end{figure}

    It can be observed that there are now some extra components added into the text image
    that aren't connected to the text blocks they are contained in.
    These are the newly added elements previously found in the non-text image.
    It can also be noticed that some elements were removed from the text image, which
    are the elements that were moved to the non-text image.
    Finally, morphological closing is applied yet again to deduce the boundaries of the text blocks,
    since the newly added text elements need to be connected to their belonging text blocks.
    The morphological closing is done both horizontally and vertically this time, in our implementation,
    a kernel of height 3 and width 3 were used and the closing was done for 4 iterations.
    It can be observed from Fig.~\ref{img:intersections-morph-close} that the new components are now
    connected with their text-blocks.

    \begin{figure}[htbp]
        \centerline{\includegraphics[width=0.5\linewidth]{img/intersections-morph-close.png}}
        \caption{Applying morphological closing on the bounding box text image with new components.}
        \label{img:intersections-morph-close}
    \end{figure}

    The text image is now finally ready to be segmented into paragraphs.
    After the text regions have been deduced, each region is separated taking in consideration
    excluding components intersected with multiple regions.
    The bounding boxes of each regions can be seen in Fig.~\ref{img:text-blocks}.

    \begin{figure}[htbp]
        \centerline{\includegraphics[width=0.5\linewidth]{img/text-blocks.png}}
        \caption{Bounding boxes of text regions.}
        \label{img:text-blocks}
    \end{figure}

    Text regions is segmented by comparing 3 lines at the same time, and when an indented starting
    or ending line is found the regions is segmented.
    This can be illustrated in Fig.~\ref{fig:paragraph-segmentation}.

    \begin{figure}[htbp]
        \centerline{\includegraphics[width=\linewidth]{img/paragraph-segmentation.jpg}}
        \caption{Example of paragraph segmentation \cite{mhs}.}
        \label{fig:paragraph-segmentation}
    \end{figure}

    When the middle index is pointing to an indented line and the next index is also pointing to one,
    the region is segmented between the middle and the next index.
    If only the middle index is pointing to an indented line, the region is split between the previous
    and the middle index.
    The purpose of the previous index here is being a reference for comparison, it is used to determine
    if a line is indented or not by comparing between their widths.
    The splits segmenting paragraphs can be seen in Fig.~\ref{img:paragraphs}.

    \begin{figure}[htbp]
        \centerline{\includegraphics[width=0.5\linewidth]{img/paragraphs.png}}
        \caption{Bounding boxes of text regions after paragraph segmentation.}
        \label{img:paragraphs}
    \end{figure}

    After this, these regions are smoothed using morphological closing, a small
    3x3 kernel with 4 iterations will suffice.
    Fig.~\ref{img:paragraphs-smoothed} shows the finalized text regions.

    \begin{figure}[htbp]
        \centerline{\includegraphics[width=0.5\linewidth]{img/bb-smoothed.png}}
        \caption{Smoothed text regions after paragraph segmentation.}
        \label{img:paragraphs-smoothed}
    \end{figure}

    \subsection{Region Refinement}

    We now have separate text and non-text documents with the text document finalized.
    Text regions has already been refined in the text segmentation step.
    We can see from Fig.~\ref{img:segmented} that text regions are finalized,
    whereas the image in the top right isn't contoured in the correct matter.

    \begin{figure}[htbp]
        \centerline{\includegraphics[width=0.5\linewidth]{img/segmented.png}}
        \caption{Text and non-text components after non-text classification.}
        \label{img:segmented}
    \end{figure}

    To refine non-text regions, for each non-text component we check if the bounding
    box of this components intersects with other non-text components, and if the
    intersection between this component and the other smaller components it intersects
    with is greater than or equals to 0.9, then remove the other smaller components and
    replace the components with its bounding box. Fig.~\ref{img:refined} shows the
    regions contours after refinement.

    \begin{figure}[htbp]
        \centerline{\includegraphics[width=0.5\linewidth]{img/refined.png}}
        \caption{Contours of refined regions.}
        \label{img:refined}
    \end{figure}

    \subsection{Non-Text Classification}

    This step classifies each non-text component into one of the following classes:
    \{image, line (horizontal or vertical), separator, negative text\}.
    The MHS originally classifies tables too, however, in our implementation table
    classification is not implemented due to its inaccuracy and tables variations
    that can't yet be effectively classified without using learning algorithms.
    
    Firstly, each component with an area less than 50 pixels is considered as noise
    and then removed.
    The rest of the components continues on.
    If the component's density is almost 1 (bigger than or equals to 0.9), then
    it is a negative text candidate.
    This candidate is then negated and then passed to the multi-layer classifier.
    If the sum of the areas of the text components is bigger than the sum of the
    areas of the non-text ones achieved from the multi-layer classifier, then it
    is indeed a negative text component.
    If the component's density is less or equals to 0.1 then this component is a line.
    The orientation of the line is determined by checking the larger dimension,
    if its width is larger that its height, then it's a horizontal line, else, it's
    a vertical line.
    If the component's density is less than or equals to 0.02 and the component
    contains text components, then it is classified as a separator.
    Otherwise, the component is an image.

    After the classification step, text and non-text regions are labeled as shown
    in Fig.~\ref{img:labeled}, providing the final page layout.

    \begin{figure}[htbp]
        \centerline{\includegraphics[width=0.5\linewidth]{img/labeled.png}}
        \caption{Labeled final page layout image.}
        \label{img:labeled}
    \end{figure}

    \section{Results and Discussions}\label{sect:3}
    
    In this section, we will show the layout analysis results of some images from
    Pattern Recognition \& Image Analysis Research Lab (PRIMA) dataset and their
    corresponding ground truths.
    This comparison is illustrated in Fig.~\ref{res:1}, Fig.~\ref{res:2} and Fig.~\ref{res:3}.

    \begin{figure*}[htbp]
        \centering

        \begin{subfigure}[b]{0.22\linewidth}
            \centering
            \includegraphics[width=\linewidth]{img/results/00000085.png}
            \caption{}
            \label{res:1:85}
        \end{subfigure}
        \hfill
        \begin{subfigure}[b]{0.22\linewidth}
            \centering
            \includegraphics[width=\linewidth]{img/results/00000088.png}
            \caption{}
            \label{res:1:88}
        \end{subfigure}
        \hfill
        \begin{subfigure}[b]{0.22\linewidth}
            \centering
            \includegraphics[width=\linewidth]{img/results/00000126.png}
            \caption{}
            \label{res:1:126}
        \end{subfigure}
        \hfill
        \begin{subfigure}[b]{0.22\linewidth}
            \centering
            \includegraphics[width=\linewidth]{img/results/00000158.png}
            \caption{}
            \label{res:1:158}
        \end{subfigure}

        \begin{subfigure}[b]{0.22\linewidth}
            \centering
            \includegraphics[width=\linewidth]{img/ground truth/00000085.png}
            \caption{}
            \label{res:1:85:gt}
        \end{subfigure}
        \hfill
        \begin{subfigure}[b]{0.22\linewidth}
            \centering
            \includegraphics[width=\linewidth]{img/ground truth/00000088.png}
            \caption{}
            \label{res:1:88:gt}
        \end{subfigure}
        \hfill
        \begin{subfigure}[b]{0.22\linewidth}
            \centering
            \includegraphics[width=\linewidth]{img/ground truth/00000126.png}
            \caption{}
            \label{res:1:126:gt}
        \end{subfigure}
        \hfill
        \begin{subfigure}[b]{0.22\linewidth}
            \centering
            \includegraphics[width=\linewidth]{img/ground truth/00000158.png}
            \caption{}
            \label{res:1:158:gt}
        \end{subfigure}

        \begin{subfigure}[b]{0.22\linewidth}
            \centering
            \includegraphics[width=\linewidth]{img/results/00000190.png}
            \caption{}
            \label{res:1:190}
        \end{subfigure}
        \hfill
        \begin{subfigure}[b]{0.22\linewidth}
            \centering
            \includegraphics[width=\linewidth]{img/results/00000194.png}
            \caption{}
            \label{res:1:194}
        \end{subfigure}
        \hfill
        \begin{subfigure}[b]{0.22\linewidth}
            \centering
            \includegraphics[width=\linewidth]{img/results/00000197.png}
            \caption{}
            \label{res:1:197}
        \end{subfigure}
        \hfill
        \begin{subfigure}[b]{0.22\linewidth}
            \centering
            \includegraphics[width=\linewidth]{img/results/00000201.png}
            \caption{}
            \label{res:1:201}
        \end{subfigure}

        \begin{subfigure}[b]{0.22\linewidth}
            \centering
            \includegraphics[width=\linewidth]{img/ground truth/00000190.png}
            \caption{}
            \label{res:1:190:gt}
        \end{subfigure}
        \hfill
        \begin{subfigure}[b]{0.22\linewidth}
            \centering
            \includegraphics[width=\linewidth]{img/ground truth/00000194.png}
            \caption{}
            \label{res:1:194:gt}
        \end{subfigure}
        \hfill
        \begin{subfigure}[b]{0.22\linewidth}
            \centering
            \includegraphics[width=\linewidth]{img/ground truth/00000197.png}
            \caption{}
            \label{res:1:197:gt}
        \end{subfigure}
        \hfill
        \begin{subfigure}[b]{0.22\linewidth}
            \centering
            \includegraphics[width=\linewidth]{img/ground truth/00000201.png}
            \caption{}
            \label{res:1:201:gt}
        \end{subfigure}

        \caption{Example 1 of results (a, b, c, d, i, j, k, l) and ground truths (e, f, g, h, m, n, o, p) comparison.}
        \label{res:1}
    \end{figure*}

    \begin{figure*}[htbp]
        \centering

        \begin{subfigure}[b]{0.22\linewidth}
            \centering
            \includegraphics[width=\linewidth]{img/results/00000213.png}
            \caption{}
            \label{res:2:213}
        \end{subfigure}
        \hfill
        \begin{subfigure}[b]{0.22\linewidth}
            \centering
            \includegraphics[width=\linewidth]{img/results/00000235.png}
            \caption{}
            \label{res:2:235}
        \end{subfigure}
        \hfill
        \begin{subfigure}[b]{0.22\linewidth}
            \centering
            \includegraphics[width=\linewidth]{img/results/00000246.png}
            \caption{}
            \label{res:2:246}
        \end{subfigure}
        \hfill
        \begin{subfigure}[b]{0.22\linewidth}
            \centering
            \includegraphics[width=\linewidth]{img/results/00000249.png}
            \caption{}
            \label{res:2:249}
        \end{subfigure}

        \begin{subfigure}[b]{0.22\linewidth}
            \centering
            \includegraphics[width=\linewidth]{img/ground truth/00000213.png}
            \caption{}
            \label{res:2:213:gt}
        \end{subfigure}
        \hfill
        \begin{subfigure}[b]{0.22\linewidth}
            \centering
            \includegraphics[width=\linewidth]{img/ground truth/00000235.png}
            \caption{}
            \label{res:2:235:gt}
        \end{subfigure}
        \hfill
        \begin{subfigure}[b]{0.22\linewidth}
            \centering
            \includegraphics[width=\linewidth]{img/ground truth/00000246.png}
            \caption{}
            \label{res:2:246:gt}
        \end{subfigure}
        \hfill
        \begin{subfigure}[b]{0.22\linewidth}
            \centering
            \includegraphics[width=\linewidth]{img/ground truth/00000249.png}
            \caption{}
            \label{res:2:249:gt}
        \end{subfigure}

        \begin{subfigure}[b]{0.22\linewidth}
            \centering
            \includegraphics[width=\linewidth]{img/results/00000263.png}
            \caption{}
            \label{res:2:263}
        \end{subfigure}
        \hfill
        \begin{subfigure}[b]{0.22\linewidth}
            \centering
            \includegraphics[width=\linewidth]{img/results/00000274.png}
            \caption{}
            \label{res:2:274}
        \end{subfigure}
        \hfill
        \begin{subfigure}[b]{0.22\linewidth}
            \centering
            \includegraphics[width=\linewidth]{img/results/00000276.png}
            \caption{}
            \label{res:2:276}
        \end{subfigure}
        \hfill
        \begin{subfigure}[b]{0.22\linewidth}
            \centering
            \includegraphics[width=\linewidth]{img/results/00000279.png}
            \caption{}
            \label{res:2:279}
        \end{subfigure}

        \begin{subfigure}[b]{0.22\linewidth}
            \centering
            \includegraphics[width=\linewidth]{img/ground truth/00000263.png}
            \caption{}
            \label{res:2:263:gt}
        \end{subfigure}
        \hfill
        \begin{subfigure}[b]{0.22\linewidth}
            \centering
            \includegraphics[width=\linewidth]{img/ground truth/00000274.png}
            \caption{}
            \label{res:2:274:gt}
        \end{subfigure}
        \hfill
        \begin{subfigure}[b]{0.22\linewidth}
            \centering
            \includegraphics[width=\linewidth]{img/ground truth/00000276.png}
            \caption{}
            \label{res:2:276:gt}
        \end{subfigure}
        \hfill
        \begin{subfigure}[b]{0.22\linewidth}
            \centering
            \includegraphics[width=\linewidth]{img/ground truth/00000279.png}
            \caption{}
            \label{res:2:279:gt}
        \end{subfigure}

        \caption{Example 2 of results (a, b, c, d, i, j, k, l) and ground truths (e, f, g, h, m, n, o, p) comparison.}
        \label{res:2}
    \end{figure*}

    \begin{figure*}[htbp]
        \centering

        \begin{subfigure}[b]{0.22\linewidth}
            \centering
            \includegraphics[width=\linewidth]{img/results/00000352.png}
            \caption{}
            \label{res:3:352}
        \end{subfigure}
        \hfill
        \begin{subfigure}[b]{0.22\linewidth}
            \centering
            \includegraphics[width=\linewidth]{img/results/00000354.png}
            \caption{}
            \label{res:3:354}
        \end{subfigure}
        \hfill
        \begin{subfigure}[b]{0.22\linewidth}
            \centering
            \includegraphics[width=\linewidth]{img/results/00000357.png}
            \caption{}
            \label{res:3:357}
        \end{subfigure}
        \hfill
        \begin{subfigure}[b]{0.22\linewidth}
            \centering
            \includegraphics[width=\linewidth]{img/results/00000358.png}
            \caption{}
            \label{res:3:358}
        \end{subfigure}

        \begin{subfigure}[b]{0.22\linewidth}
            \centering
            \includegraphics[width=\linewidth]{img/ground truth/00000352.png}
            \caption{}
            \label{res:3:352:gt}
        \end{subfigure}
        \hfill
        \begin{subfigure}[b]{0.22\linewidth}
            \centering
            \includegraphics[width=\linewidth]{img/ground truth/00000354.png}
            \caption{}
            \label{res:3:354:gt}
        \end{subfigure}
        \hfill
        \begin{subfigure}[b]{0.22\linewidth}
            \centering
            \includegraphics[width=\linewidth]{img/ground truth/00000357.png}
            \caption{}
            \label{res:3:357:gt}
        \end{subfigure}
        \hfill
        \begin{subfigure}[b]{0.22\linewidth}
            \centering
            \includegraphics[width=\linewidth]{img/ground truth/00000358.png}
            \caption{}
            \label{res:3:358:gt}
        \end{subfigure}

        \begin{subfigure}[b]{0.22\linewidth}
            \centering
            \includegraphics[width=\linewidth]{img/results/00000359.png}
            \caption{}
            \label{res:3:359}
        \end{subfigure}
        \hfill
        \begin{subfigure}[b]{0.22\linewidth}
            \centering
            \includegraphics[width=\linewidth]{img/results/00000396.png}
            \caption{}
            \label{res:3:396}
        \end{subfigure}
        \hfill
        \begin{subfigure}[b]{0.22\linewidth}
            \centering
            \includegraphics[width=\linewidth]{img/results/00000407.png}
            \caption{}
            \label{res:3:407}
        \end{subfigure}
        \hfill
        \begin{subfigure}[b]{0.22\linewidth}
            \centering
            \includegraphics[width=\linewidth]{img/results/00000880.png}
            \caption{}
            \label{res:3:880}
        \end{subfigure}

        \begin{subfigure}[b]{0.22\linewidth}
            \centering
            \includegraphics[width=\linewidth]{img/ground truth/00000359.png}
            \caption{}
            \label{res:3:359:gt}
        \end{subfigure}
        \hfill
        \begin{subfigure}[b]{0.22\linewidth}
            \centering
            \includegraphics[width=\linewidth]{img/ground truth/00000396.png}
            \caption{}
            \label{res:3:396:gt}
        \end{subfigure}
        \hfill
        \begin{subfigure}[b]{0.22\linewidth}
            \centering
            \includegraphics[width=\linewidth]{img/ground truth/00000407.png}
            \caption{}
            \label{res:3:407:gt}
        \end{subfigure}
        \hfill
        \begin{subfigure}[b]{0.22\linewidth}
            \centering
            \includegraphics[width=\linewidth]{img/ground truth/00000880.png}
            \caption{}
            \label{res:3:880:gt}
        \end{subfigure}

        \caption{Example 3 of results (a, b, c, d, i, j, k, l) and ground truths (e, f, g, h, m, n, o, p) comparison.}
        \label{res:3}
    \end{figure*}

    \section{Limitations}\label{sect:4}
    
    The performance of our implementation is reasonable.
    However it lacks in accuracy.
    Even though the results looks promising, it still needs improvements in several areas.
    For instance, the binary image quality needs to be improved substantially in order
    to classify components correctly.
    The segmentation process fails hardly for some images, see Fig.~\ref{res:1:201}.
    In addition, text segmentation still needs improvements, due to the existence of
    under-segmented areas where text blocks are very close to each other and
    over-segmented areas where text text lines and/or words are very far from each other.
    Future work might start by improving these areas.
    Moreover, the classification of non-text elements is still lacking in identifying
    tables, headings, headers, footers, graphics, graphs, captions, credits, floating text, drop-capitals
    and others, which requires great effort to implement.
    Furthermore, some miss-classification exists in non-text components, some dark images are classified as
    negative text, where they really are just normal images.
    Further, negative text is very general, headers and headings for example, can be negative text, hence,
    identifying the type of the negative text block is important.
    These are less crucial than the previous faults concerning segmentation, but still worth improving.

    \section{Conclusion}\label{sect:5}

    To sum up, Document layout analysis is a difficult process that can only produce sufficient outcomes through several
    levels of processing and computations, as can be seen from the implementation of this system.
    The layers in the MHS system are based on carefully extracted properties from many documents with different layouts.
    However, this system is heavily dependant on the quality of the binary image.
    In addition, the processes in this system are heavily dependant on statistics and observations found in most
    documents, which determines the accuracy of the system.
    Nonetheless, it still has room for some improvements.

    \bibliographystyle{IEEEtran}
    \bibliography{IEEEabrv, references}

\end{document}